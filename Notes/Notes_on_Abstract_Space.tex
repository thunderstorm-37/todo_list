%\documentclass[12pt,a4paper]{article}  % Use this line if this document will be released
\documentclass[12pt,a4paper,draft]{article}  % Use this line if this document is a draft
\usepackage{ifdraft}


%% Bibliography
\usepackage{etoolbox}
\newcommand{\bibfile}{\jobname.bib}  % Name of the BibTeX file.
% ref.bib should be a symbolic link to the universal BibTeX file, which should be a local copy of
% https://github.com/equipez/bibliographie/blob/main/ref.bib
% Run `getbib` in the current directory under the draft mode to get the BibTeX file containing only
% the cited references. The name will be xyz.bib if this TeX file is xyz.tex.
\newcommand{\universalbib}{ref.bib}
\ifdraft{\IfFileExists{\universalbib}{\renewcommand{\bibfile}{\universalbib}}{}}{}
% The counter `cite' is used to count the number of citations.
\newcounter{cite}
\pretocmd{\cite}{\stepcounter{cite}}{}{}


%% Add line numbers in draft mode
\RequirePackage[mathlines]{lineno}
\ifdraft{\linenumbers}{}
\renewcommand{\linenumberfont}{\normalfont\scriptsize\sffamily\color{gray}}
\setlength{\linenumbersep}{\marginparsep}


%% Geometry
%\voffset=-1.5cm \hoffset=-1.4cm \textwidth=16cm \textheight=22.0cm  % Luis' setting
\usepackage[a4paper, textwidth=16.0cm, textheight=22.0cm]{geometry}
\renewcommand{\baselinestretch}{1.2}


%% Basic packages
\usepackage{amsmath,amsthm,amssymb,amsfonts}
\usepackage{mathtools}  % Provides \coloneqq
\usepackage{empheq}
\usepackage{xcolor}
\usepackage[bbgreekl]{mathbbol}
\DeclareSymbolFontAlphabet{\mathbbm}{bbold}
\DeclareSymbolFontAlphabet{\mathbb}{AMSb}
\usepackage{bbm}
\usepackage{upgreek}
\usepackage{accents}
\usepackage{xspace}
\usepackage{rotating}
\usepackage{multirow,booktabs}
\usepackage[en-US]{datetime2}


%% Format of the table of content
\usepackage[normalem]{ulem}
\usepackage[toc,page]{appendix}
\renewcommand{\appendixpagename}{\Large{Appendix}}
\renewcommand{\appendixname}{Appendix}
\renewcommand{\appendixtocname}{Appendix}
%\usepackage{sectsty}
\setcounter{tocdepth}{2}


%% Section title style
\usepackage{sectsty}
\sectionfont{\large}
\subsectionfont{\large}


%% Some colors
\definecolor{darkblue}{rgb}{0,0.1,0.5}
\definecolor{darkgreen}{rgb}{0,0.5,0.1}
\definecolor{darkyellow}{rgb}{0.65,0.65,0.01}


%% Todo notes
\ifdraft{
    \setlength{\marginparwidth}{2.42cm}
    \usepackage[tickmarkheight=3pt,textsize=small,backgroundcolor=blue!16,linecolor=purple,bordercolor=purple]{todonotes}
}{
    \newcommand{\todo}[1]{}
    \newcommand{\listoftodos}{}
}


%% Graph, tikz and pgf
%\usepackage{subfigure}
\setlength{\unitlength}{1mm}
% The \unitlength command is a Length command. It defines the units used in the Picture Environment.
\usepackage{graphicx}
%\usepackage{tikz,tikzscale,pgf,pgfarrows,pgfnodes,filecontents,tikz-cd}
\usepackage{tikz,tikzscale,pgf}
\usetikzlibrary{arrows,arrows.meta,patterns,positioning,decorations.markings,shapes}
\usepackage{pgfplots}
\usepackage{pgfplotstable}
\usepackage[justification=centering]{caption}
\usepgfplotslibrary{fillbetween}
\pgfplotsset{compat=1.11}


%% Turn off some unharmful warnings in draft mode
%% N.B.: DO NOT use `silence` together with `hyperref`. They will cause an infinite loop.
\ifdraft{
    \usepackage{silence}
    \WarningFilter{xcolor}{Incompatible color definition on}
    \WarningFilter{hyperref}{Draft mode on}
    \WarningFilter{refcheck}{Unused label}
    \WarningFilter{microtype}{`draft' option active}
    \WarningFilter{latex}{Writing or overwriting file} % Mute the warning about 'writing/overwriting file'
    \WarningFilter{latex}{Writing file} % Mute the warning about 'writing/overwriting file'
    \WarningFilter{latex}{Tab has} % Mute the warning about 'Tab has been converted to Blank Space'
    \WarningFilter{latex}{Marginpar on page} % Mute the warning about 'Marginpar on page xx moved'
    \WarningFilter{latex}{author given} % Mute the warning about 'No \author given'
}{}


%% Hyperref, url, and email
%% N.B.: DO NOT use `silence` together with `hyperref`. They will cause an infinite loop.
\ifdraft{\usepackage{refcheck}\newcommand{\url}{\texttt}}{
    \usepackage{hyperref}
    \hypersetup{colorlinks, linkcolor=darkblue, anchorcolor=darkblue, citecolor=darkblue, urlcolor=darkblue}
    \usepackage{url}
} % Check unused labels
\newcommand{\email}{\texttt}


%% Enumerate and itemize
\usepackage{eqlist}
\usepackage{enumitem}
\setlist[itemize]{leftmargin=*}
\setlist[enumerate]{leftmargin=*,label=\normalfont{(\alph*)}}


%% Algorithm environment
\usepackage[section]{algorithm}
\usepackage{algpseudocode,algorithmicx}
\newcommand{\INPUT}{\textbf{Input}}
\newcommand{\FOR}{\textbf{For}~}
\algrenewcommand\algorithmicrequire{\textbf{Input:}}
\algrenewcommand\algorithmicensure{\textbf{Output:}}
\algrenewcommand\alglinenumber[1]{\normalsize #1.}
\newcommand*\Let[2]{\State #1 $=$ #2}


%% Theorem-like environments
\newtheorem{theorem}{Theorem}[section]
\newtheorem{conjecture}{Conjecture}[section]
\newtheorem{corollary}{Corollary}[section]
\newtheorem{exercise}{Exercise}[section]
\newtheorem{lemma}{Lemma}[section]
\newtheorem{problem}{Problem}[section]
\newtheorem{proposition}{Proposition}[section]
\newtheorem{assumption}{Assumption}[section]
\newtheorem{example}{Example}[section]
\newtheorem{question}{Question}[section]
% Change theoremstyle to ``definition'', which uses textnormal for the text.
\theoremstyle{definition}
\newtheorem{definition}{Definition}[section]
\newtheorem{remark}{Remark}[section]
% proof
\usepackage{xpatch}
\xpatchcmd{\proof}{\itshape}{\normalfont\proofnamefont}{}{}
\newcommand{\proofnamefont}{\bfseries}

%% Equation numbering
\numberwithin{equation}{section}


%% Fine tuning
\usepackage{microtype}
\usepackage[nobottomtitles*]{titlesec} % No section title at the bottom of pages
% Prevent footnote from running to the next page
\interfootnotelinepenalty=10000
% No line break in inline math
\interdisplaylinepenalty=10000
\relpenalty=10000
\binoppenalty=10000
% No widow or orphan lines
\clubpenalty=10000
\widowpenalty=10000
\displaywidowpenalty=10000


% Use @ to put 1 math unit (mu) in math
% See https://nhigham.com/2013/01/07/fine-tuning-spacing-in-latex-equations/
% and also TeXbook p. 155.
\mathcode`@="8000{\catcode`\@=\active\gdef@{\mkern1mu}}


%% Operators, commands
\usepackage{relsize}
\usepackage{nccmath}
%\DeclareMathOperator*{\mcap}{\,\medmath{\bigcap}\,}
%\DeclareMathOperator*{\mcup}{\,\medmath{\bigcup}\,}
\DeclareMathOperator*{\mcap}{\,\mathsmaller{\bigcap}\,}
\DeclareMathOperator*{\mcup}{\,\mathsmaller{\bigcup}\,}
%\renewcommand{\cap}{\mcap}
%\renewcommand{\cup}{\mcup}

\newcommand{\ceil}[1]{ {\lceil{#1}\rceil} }
\newcommand{\floor}[1]{ {\lfloor{#1}\rfloor} }

\DeclareMathOperator{\tr}{tr}
\DeclareMathOperator{\sort}{sort}
\DeclareMathOperator*{\Argmax}{Argmax}
\DeclareMathOperator*{\Argmin}{Argmin}
\DeclareMathOperator*{\Arglocmin}{Arglocmin}
\DeclareMathOperator*{\argmax}{argmax}
\DeclareMathOperator*{\argmin}{argmin}
\DeclareMathOperator*{\diag}{diag}
\DeclareMathOperator*{\Diag}{Diag}
\DeclareMathOperator{\Span}{span}
\DeclareMathOperator{\med}{med}
\DeclareMathOperator{\essinf}{essinf}
\DeclareMathOperator{\cl}{cl}
\DeclareMathOperator{\vol}{vol}
\DeclareMathOperator{\comp}{C}
\DeclareMathOperator{\sign}{sign}
\DeclareMathOperator{\rank}{rank}
\DeclareMathOperator{\range}{range}
\DeclareMathOperator{\card}{card}
\DeclareMathOperator{\diam}{diam}
\DeclareMathOperator{\dist}{dist}
\newcommand{\disth}{{\operatorname{\updelta_{\sss{H}}}}}
\newcommand{\ind}{\mathbbm{1}}
%\newcommand*{\defeq}{\stackrel{\mbox{\normalfont\tiny{\textnormal{def}}}}{=}}
\newcommand\defeq{\mathrel{\overset{\makebox[0pt]{\mbox{\normalfont\tiny\sffamily def}}}{=}}}

\newcommand{\RR}{\mathbb{R}}
\newcommand{\BB}{\mathcal{B}}
\renewcommand{\SS}{\mathbb{S}}
\newcommand{\TT}{\mathcal{T}}
\newcommand{\ZZ}{\mathbb{Z}}
\newcommand{\NN}{\mathbb{N}}
\newcommand{\FF}{\mathcal{F}}
\newcommand{\CC}{\mathbb{C}}
\newcommand{\XX}{\mathcal{X}}
\newcommand{\sset}{\mathcal{S}}
\newcommand{\pen}{h}
\newcommand{\penpar}{\mu}
\newcommand{\res}{\rho}
\newcommand{\col}{r}
\newcommand{\ofd}{\mathcal{F}}
\newcommand{\stf}[1]{\mathbb{S}^{#1}}
\newcommand{\sss}[1]{{\scriptscriptstyle{#1}}}
\newcommand{\sK}{{\scriptscriptstyle{K}}}
\newcommand{\sT}{{\scriptscriptstyle{T}}}
\newcommand{\fro}{{\scriptstyle{\textnormal{F}}}}
\newcommand{\trs}{{\scriptstyle{\mathsf{T}}}}
\newcommand{\hmt}{{\scriptstyle{{\mathsf{H}}}}}
\newcommand{\pin}{{\scriptstyle{{\mathsf{+}}}}}
\newcommand{\inv}{{-1}}
\newcommand{\adj}{*}
\newcommand{\ones}{\mathbf{1}}

\newcommand{\cs}{\text{c}}
\newcommand{\hp}{\circ}
\newcommand{\cc}{\sss{\textnormal{C}}}
\newcommand{\dec}{\sss{\textnormal{D}}}
\newcommand{\cauchy}{\sss{\textnormal{C}}}
\newcommand{\scauchy}{\sss{\textnormal{S}}}
\newcommand{\crit}{\textnormal{crit}}
\newcommand{\rsg}{\hat{\partial}}
\newcommand{\gsg}{\partial}
\newcommand{\dom}{\textnormal{dom}}
\newcommand{\tf}{{\textnormal{f}}}
\newcommand{\tg}{{\textnormal{g}}}
\newcommand{\ts}{{\textnormal{s}}}
\newcommand{\st}{\textnormal{s.t.}}
\newcommand{\etc}{{etc.}\xspace}
\newcommand{\ie}{{i.e.}\xspace}
\newcommand{\eg}{{e.g.}\xspace}
\newcommand{\etal}{{et al.}\xspace}
\newcommand{\iid}{\text{i.i.d.}\xspace}
\newcommand{\as}{\text{a.s.}\xspace}

\newcommand{\me}{\mathrm{e}}
\newcommand{\md}{\mathrm{d}}
\newcommand{\mi}{\mathrm{i}}
\newcommand{\lev}{\mathrm{lev}}
\newcommand{\bA}{\mathbf{A}}
\newcommand{\bx}{\mathbf{u}}
%\newcommand{\bb}{\mathbf{f}}
\newcommand{\bb}{\mathbf{r}}
\newcommand{\nov}{n_{\textnormal{o}}}
\xspaceaddexceptions{]\}}
% tex.stackexchange.com/questions/15252/why-does-xspace-behave-differently-for-parenthesis-vs-braces-brackets
\newcommand{\MATLAB}{\textsc{Matlab}\xspace}
\newcommand{\octave}{\mbox{GNU Octave}\xspace}
\newcommand{\prblm}{\texttt}
\DeclareMathAlphabet{\mathsfit}{T1}{\sfdefault}{\mddefault}{\sldefault}
\SetMathAlphabet{\mathsfit}{bold}{T1}{\sfdefault}{\bfdefault}{\sldefault}
\newcommand{\prbb}{\mathsfit{p}}
\newcommand{\pp}{\mathsf{p}}
\newcommand{\qq}{\mathsf{q}}
\newcommand{\ttt}{\mathsfit{t}}
\newcommand{\tol}{\varepsilon}
\newcommand{\bt}{\mathbf{t}}
\newcommand{\br}{\mathbf{r}}
\newcommand{\dd}{\mathbf{d}}
\newcommand{\ii}{\mathbf{i}}
\newcommand{\jj}{\mathbf{j}}
\newcommand{\xx}{\mathbf{x}}
\renewcommand{\pp}{\mathbf{p}}
\renewcommand{\ggg}{\mathbf{g}}
\newcommand{\GG}{\mathbf{G}}
\DeclareMathOperator{\expc}{\mathbb{E}}
\renewcommand{\Pr}{\mathbb{P}}
\newcommand{\lb}{\underline}
\newcommand{\ub}{\overline}

% mathlcal font
\DeclareFontFamily{U}{dutchcal}{\skewchar\font=45 }
\DeclareFontShape{U}{dutchcal}{m}{n}{<-> s*[1.0] dutchcal-r}{}
\DeclareFontShape{U}{dutchcal}{b}{n}{<-> s*[1.0] dutchcal-b}{}
\DeclareMathAlphabet{\mathlcal}{U}{dutchcal}{m}{n}
\SetMathAlphabet{\mathlcal}{bold}{U}{dutchcal}{b}{n}

% mathscr font (supporting lowercase letters)
%\usepackage[scr=dutchcal]{mathalfa}
%\usepackage[scr=esstix]{mathalfa}
%\usepackage[scr=boondox]{mathalfa}
%\usepackage[scr=boondoxo]{mathalfa}
\usepackage[scr=boondoxupr]{mathalfa}
%\newcommand{\model}{\mathscr{h}}
\newcommand{\model}{\tilde{f}}
\newcommand{\rmod}{F}

\newcommand{\Set}[1]{\mathcal{#1}}
\DeclareMathAlphabet{\mathpzc}{OT1}{pzc}{m}{it} % The mathpzc font
\newcommand{\slv}{\mathpzc}
% mathpzc looks great, but it stops working on 19 Feb 2020 for no reason.
%\newcommand{\slv}{\mathscr}
\newcommand{\software}{\texttt}
\DeclareMathOperator{\eff}{\mathsf{e}\;\!}
%\DeclareMathOperator{\Eff}{\mathsf{E}\;\!}
\newcommand{\out}{{\text{out}}}


%% Commands for revision
\newcommand{\red}[1]{\textcolor{red}{#1}}
\newcommand{\blue}[1]{\textcolor{blue}{#1}}
\newcommand{\green}[1]{\textcolor{darkgreen}{#1}}
\newcommand{\TYPO}[1]{{\color{orange}{#1}}}
\newcommand{\MISTAKE}[1]{{\color{violet}{#1}}}
%\newcommand{\REPHRASE}[1]{{\color{darkgreen}{#1}}}
\newcommand{\REVISE}[1]{{\color{blue}{#1}}}
\newcommand{\REVISEred}[1]{{\color{red}{#1}}}
\newcommand{\COMMENT}{\todo}  % Needs the todonotes package
%\newcommand{\COMMENT}[1]{\textcolor{brown}{{\small{(comment: #1)}}}}  % This puts comments inline

% Use the following if revision is finished
%\%newcommand{\TYPO}{}
%\newcommand{\MISTAKE}{}
%\newcommand{\REPHRASE}{}
%\newcommand{\REVISE}{}
%\newcommand{\REVISEred}{}
%\newcommand{\COMMENT}[1]{}  % Input ignored.

\usepackage{xfrac}

%%%%%%%%%%%%%%%%%%%%%%%%%%%%%%%%%%%%%%%%%%%%%%%%%%%%%%%%%%%%%%%%%%%%%%%%%%%%%%%%%%%%%%%%%%%%%%%%%%%%
\title{Notes on Abstract Space}

\date{\DTMnow}

\author{
    Zhu Huatao
    %\thanks{}
    %\and
    %Author2
    %\thanks{Information2}
}


\begin{document}

\maketitle

%\begin{abstract}
%\end{abstract}

%\textbf{Keywords}: Keyword1, Keyword2
%%%%%%%%%%%%%%%%%%%%%%%%%%%%%%%%%%%%%%%%%%%%%%%%%%%%%%%%%%%%%%%%%%%%%%%%%%%%%%%%%%%%%%%%%%%%%%%%%%%%

These are notes on abstract spaces, mainly based on the book \textit{An Introduction to Abstract Spaces} by Hu Shigeng and Zhang Xianwen.

\section{Basic Concepts}

\subsection{Sets and Relations}

We denote by $2^X$ the power set of $X$, i.e., the set of all subsets of $X$. Some related notation is as follows. For any $\mathcal{A} \subset 2^X$, we set
$$ \bigcup \mathcal{A} = \bigcup\{A:A\in \mathcal{A}\}, $$
and
$$ \bigcap \mathcal{A} = \bigcap\{A:A\in \mathcal{A}\}. $$
In addition, we define
$$ \mathcal{A}^* = \{\bigcap \mathcal{B}: \mathcal{B} \subset \mathcal{A} \text{ and } |\mathcal{B}| \text{ is finite} \}.
$$
If $\mathcal{A},\mathcal{B} \subset 2^X$, we write
$$ \mathcal{A} \vdash \mathcal{B} \iff \forall B \in \mathcal{B},\ \exists A \in \mathcal{A}:\ A \subset B,$$
and
$$ \mathcal{A} \prec \mathcal{B} \iff \forall A \in \mathcal{A},\ \exists B \in \mathcal{B}:\ A \subset B.
$$

\begin{definition}~\label{def1.1}
    If $\mathcal{A} \subset 2^X$ is nonempty and satisfies
    \begin{enumerate}
        \item $\emptyset\notin \mathcal{A}$;
        \item For any $A_1,A_2 \in \mathcal{A}$, $A_1 \cap A_2 \in \mathcal{A}$;
        \item If $\mathcal{A} \vdash A \subset X$, then $A \in \mathcal{A}$,
    \end{enumerate}
    then we call $\mathcal{A}$ a \textbf{filter} on $X$. Moreover, 
    if $\mathcal{B} \subset \mathcal{A}$ satisfies $\mathcal{B} \vdash \mathcal{A}$, 
    then we say that $\mathcal{B}$ is a \textbf{base} of the filter $\mathcal{A}$.
\end{definition}
It's easy to see that condition (b) in Definition~\ref{def1.1} can be replaced by $\mathcal{A}=\mathcal{A}^*$.

Any filter has at least one base, for example, itself. Conversely, any nonempty $\mathcal{B} \subset 2^X$ satisfying suitable conditions is a base of a filter.

\begin{theorem}
    If a nonempty $\mathcal{B} \subset 2^X$ satisfies
    \begin{enumerate}
        \item $\emptyset \notin \mathcal{B}$;
        \item For any $B_1,B_2 \in \mathcal{B}$, we have $\mathcal{B} \vdash B_1 \cap B_2$, 
    \end{enumerate}
    then $\mathcal{B}$ is a base of a filter $\mathcal{A} \subset 2^X$.
\end{theorem}   
\begin{proof}
    Let $$ \mathcal{A} = \{A \subset X: \mathcal{B} \vdash A \}. $$
    It is straightforward to verify that $\mathcal{A}$ is a filter and that $\mathcal{B}$ is a base of $\mathcal{A}$.
\end{proof}

Next, we discuss relations between two nonempty sets.
For any two nonempty sets $X$ and $Y$, we consider elements $x\in X$ and $y\in Y$ as abstract variables, and define how $x$ is related to $y$.
\begin{definition}~\label{def1.3}
    Let $X$ and $Y$ be two nonempty sets. Any subset $F \subset X \times Y$ 
    is called a \textbf{relation} from $X$ to $Y$. When $(x,y) \in F$, we say that
    $x$ is related to $y$ by the relation~$F$, denoted by $xFy$ or $y\in Fx$.
    If $F \subset X \times X$, we call it a \textbf{binary relation} on $X$. 
\end{definition}

For any relation $F$ from $X$ to $Y$ and any subset $A \subset X$, define
$$ F(A) = \{y \in Y: \exists x \in A,\ (x,y) \in F \}. $$
Let $Fx=F\{x\}$. Then $F(A) = \bigcup_{x \in A} Fx$. 
If $G\subset Y \times Z$ is another relation, the composition $G\circ F$ is defined by
$$ G \circ F = \{(x,z) \in X \times Z: \exists y \in Y,\ (x,y) \in F \text{ and } (y,z) \in G \}. $$
The inverse relation of $F$ is
$$ F^{-1} = \{(y,x) \in Y \times X: (x,y) \in F \}. $$

Let us consider any relation $F\subset X\times X$.
If $F^{-1}=F$, we say that $F$ is \textbf{symmetric}. If $F\circ F \subset F$, we say that $F$ is \textbf{transitive}.
If the diagonal $\{(x,x):x\in X\}$ of $X \times X$ is contained in $F$, we say that $F$ is \textbf{reflexive}.
If $F$ is symmetric, transitive, and reflexive, we say that $F$ is an \textbf{equivalence relation} on $X$.

Fix a relation $F\subset X\times Y$. We can view $F$ as the correspondence
$$ F:X \rightarrow 2^Y, \quad x \mapsto Fx. $$
Thus $F$ is a set-valued function. In applications, we often consider the special case that for each $x\in X$, the set $Fx$ contains exactly one element. In that case we say $F$ is a \textbf{function} from $X$ to $Y$.

\section{Abstract Space}

\subsection{Uniform Space}
The definition of uniform space is similar to that of topological space. The difference is that in a uniform space, the neighborhoods of points are defined by relations instead of subsets.
\begin{definition}~\label{def2.1}   
    Let $X$ be a nonempty set. A nonempty family $\mathcal{U}$ of relations on $X$ is called a \textbf{uniform structure} on $X$ if it satisfies the following conditions:
    \begin{enumerate}
        \item For any $U \in \mathcal{U}$, $\Delta_X \subset U$, where $\Delta_X = \{(x,x): x \in X\}$ is the diagonal of $X \times X$;
        \item If $U \in \mathcal{U}$, then $U^{-1} \in \mathcal{U}$;
        \item For any $U \in \mathcal{U}$, there exists $V \in \mathcal{U}$ such that $V \circ V \subset U$;
        \item $\mathcal{U}$ is a filter on $X \times X$.
    \end{enumerate}
    The space $X$ or the pair $(X,\mathcal{U})$ is called a \textbf{uniform space}.
\end{definition}
For any base of the filter $\mathcal{U}$, we call it a \textbf{base} of the uniform structure $(X,\mathcal{U})$.
For any $U \in \mathcal{U}$, $U \cap U^{-1} = (U \cap U^{-1})^{-1} \in \mathcal{U}$.
Moreover, we have an element $V \in \mathcal{U}$ such that $V \circ V \subset U$. 
By the fact that $V=V \circ \Delta \subset V \circ V  $, 
we have $V \circ V \in \mathcal{U}$. Thus the set~$\{U \cap U^{-1}:U \in \mathcal{U}\}$ 
or $\{V \circ V:V \in \mathcal{U}\}$ either forms a base of $\mathcal{U}$.

The condition (b) in Definition~\eqref{def2.1} means that for any $x,y \in X$,
\[
x=y \iff (x,y) \in U \text{ for all } U \in \mathcal{U}.
\]
We can say that the more $U \in \mathcal{U}$ "small", the closer $x$ and $y$ are.
The uniform structure $\mathcal{U}$ provides a set of criteria that measure the closeness between two points in the space $X$.
Those facts naturally lead to the definition of uniform neighborhood, i.e.,
\[
    \mathcal{N}_x = \{U(x): U \in \mathcal{U}\}, \quad \text{for any } x \in X.
\]
Next, we are going to prove that, there exists a unique topology $\tau$ on $X$
such that for any~$x \in X$, $\mathcal{N}_x$ is the neighborhood system of $x$ with respect to the topology $\tau$.
We call~$\tau$ the \textbf{uniform topology} induced by the uniform structure $\mathcal{U}$.
\begin{theorem} \label{thm2.1}
    Suppose that $(X,\mathcal{U})$ is a uniform space. Then the following conclusions holds:
    \begin{enumerate}
        \item There exists a unique hausdorff topology $\tau$ on $X$ such that for any $x \in X$,
            $\mathcal{N}_x$ is the neighborhood system of $x$ with respect to $\tau$.
        \item Suppose that $\mathcal{B} \subset \mathcal{U}$ is a base of $\mathcal{U}$. Define
            $$ \mathcal{B}_x = \{B(x): B \in \mathcal{B}\}. $$
            Then for any $x \in X$, $\mathcal{B}_x$ is a base of the neighborhood system $\mathcal{N}_x$ with respect to the uniform topology $\tau$.
            Thus if $\mathcal{B}$ is countable, the uniform topology $\tau$ is first countable.
        \item For any $A \in X$ and $M \in X \times X$, we have the following closure formulas with respect to the uniform topology $\tau$:
            $$ \bar{A}=\bigcap_{U \in \mathcal{U}} U(A) $$
            and 
            $$ \bar{M} = \bigcap_{U \in \mathcal{U}} (U \circ M \circ U). $$
            The uniform structure $\mathcal{U}$ can be replaced by any base of $\mathcal{U}$.
    \end{enumerate}
\end{theorem}
\begin{proof}
    \begin{enumerate}
        \item define
            $$ \tau = \{A \subset X: \forall x \in A,\ \exists U \in \mathcal{N}_x,\ U \subset A\}. $$
            First we prove that $\tau$ is a topology on $X$.
            We have that~$X,\emptyset \in \tau$. 
            For any~$A,B \in \tau$, we have~$A \cap B \in \tau$. In fact, for any~$x \in A \cap B$, 
            there exist~$U,V \in \mathcal{N}_x$ such that~$U \subset A$ and~$V \subset B$. 
            Then~$U \cap V \in \mathcal{N}_x$ and~$U \cap V \subset A \cap B$. 
            Finally, for any family~$\{A_i\}_{i \in I} \subset \tau$, we have~$\bigcup_{i \in I} A_i \in \tau$. 
            In fact, for any~$x \in \bigcup_{i \in I} A_i$, there exists~$j \in I$ such that~$x \in A_j$. 
            Since~$A_j \in \tau$, there exists~$U \in \mathcal{N}_x$ such that~$U \subset A_j \subset \bigcup_{i \in I} A_i$. 
            Thus~$\tau$ is a topology on~$X$.

            Next we prove that for any~$x \in X$,~$\mathcal{N}_x$ is the neighborhood system of~$x$ with respect to the topology~$\tau$.
            Let $\mathcal{A}_x$ be the neighborhood system of $x$ with respect to $\tau$.
            For any $U \in \mathcal{N}_x$, by the definition of $\tau$, we have $U \in \mathcal{A}_x$ and thus $\mathcal{N}_x \subset \mathcal{A}_x$.
            Conversely, for any $A \in \mathcal{A}_x$, by the definition of $\tau$, there exists $U \in \mathcal{N}_x$ such that $U \subset A$.
            Thus we have~$A \in \mathcal{N}_x$ and $\mathcal{A}_x \subset \mathcal{N}_x$.
            Therefore, $\mathcal{N}_x$ is the neighborhood system of $x$ with respect to the topology $\tau$.

            Finally, we prove that the topology $\tau$ is hausdorff.
            For any two distinct points~$x,y \in X$, there exists $U \in \mathcal{U}$ such that $(x,y) \notin U$.
            Then $y \notin U(x)$. Since $U(x) \in \mathcal{N}_x$,  then~$U(x)$ is a neighborhood of $x$ not containing $y$.
            Similarly, there exists a neighborhood of $y$ not containing $x$.
        \item It's straightforward by conclusion (a) of Theorem~\eqref{thm2.1}.
        \item By the definition of closure, we have
            $$ x \in \bar{A} \iff \forall U \in \mathcal{U},\ U(x) \cap A \neq \emptyset, $$
            which is equivalent to
            $$ \bar{A} = \bigcap_{U \in \mathcal{U}} U(A). $$
            The second closure formula can be proved similarly.
    \end{enumerate}
\end{proof}
From the construction of the neighborhood system $\mathcal{N}_x$, 
we can see that the local structure of a uniform space is uniform everywhere.
In other words, for any two points~$x,y \in X$, there exists a natural correspondence between the neighborhood systems $\mathcal{N}_x$ and $\mathcal{N}_y$.

As we have mentionded that the uniform structure $\mathcal{U}$ provides a set of criteria that measure the closeness between two points in the space $X$,
we hope to use some numerical values to describe the closeness.
Basicaly, we expect that there exists a metric that induces the uniform structure $\mathcal{U}$.
However, this is not always possible.
\begin{theorem} \label{thm2.2}
    Let $(X,\mathcal{U})$ be a uniform space. 
    There exists a metric $d$ on $X$ inducing the uniform structure $\mathcal{U}$ if and only if 
    $\mathcal{U}$ has a countable base.
\end{theorem}
\begin{proof}
    The only if part is trivial. We only prove the if part.
    Suppose that $\mathcal{U}$ has a countable base $\{U_n:n \in \NN\}$.
    Without loss of generality, we can assume that $U_n$ is symmetric and
    $$ U_n \circ U_n \circ U_n \subset U_{n-1} \quad \text{for all } n \geq 1, $$
    where we set $U_0 = X \times X$.
    For any $x,y \in X$, define
    $$ f(x,y)=
    \begin{cases}
        2^{-n}, & \text{if } (x,y) \in U_n \text{ and } (x,y) \notin U_{n+1} \text{ for some } n \in \NN, \\
        0, & \text{if } x=y,
    \end{cases}  $$
    and 
    $$ d(x,y) = \inf\left\{\sum_{i=1}^m f(x_{i-1},x_i): m \in \NN, x_0=x, x_m=y, x_i \in X \text{ for } i=1,\ldots,m-1 \right\}. $$
    For simplicity, we denote the sum in the above formula by $S(x,y)$.
    We first prove that $d$ is a metric on $X$.
    By the symmetric property of $U_n$, we know that $f$ and $d$ are symmetric.
    It's obvious that $d(x,y)=0$ is equivalent to $x=y$.
    For any $x,y,z \in X$, we have
    \begin{align}
        d(x,y) & = \inf S(x,y) \nonumber \\
        & \leq \inf(S(x,z)+S(z,y)) \nonumber \\
        & = d(x,z) + d(z,y). \nonumber
    \end{align}
    Thus $d$ is a metric on $X$.
    Then we prove that the metric $d$ induces the uniform structure~$\mathcal{U}$.
    Define 
    $$ V_r = \{(x,y) \in X \times X: d(x,y) < r\}, \quad \text{for any } r > 0. $$
    We only need to prove that 
    $$ V_{\sfrac{1}{2^{n+1}}} \subset U_n \subset V_{\sfrac{1}{2^{n-1}}} \text{ for all } n \in \NN. $$
    By the definition of $d$, it's straightforward to see that
    $ U_n \subset V_{\sfrac{1}{2^{n-1}}}. $
    Then we prove that $f(x,y) \leq 2d(x,y)$.
    We fix a finite sequence $\{x_i\}_{i=0}^m$ with $x_0=x$ and $x_m=y$ 
    and let~$a = \sum_{i=1}^m f(x_{i-1},x_i)$. 
    We are going to show that $f(x,y) \leq 2a$ by induction on $n$.
    We choose $m \in \NN$ such that $2^{-m} \leq a < 2^{-m+1}$.
    There exists the largest integer $k$ such that $\sum_{i=1}^{k-1} f(x_{i-1},x_i) \leq \frac{a}{2}$.
    Thus we know 
    $$\sum_{i=k+1}^m f(x_{i-1},x_i) = a-\sum_{i=1}^{k} f(x_{i-1},x_i) < \frac{a}{2}.$$
    By the induction hypothesis, we have 
    $$ f(x,x_{k-1}) \leq 2\sum_{i=1}^{k-1} f(x_{i-1},x_i) \leq a, $$
    and
    $$ f(x_{k},y) \leq 2\sum_{i=k+1}^{m} f(x_{i-1},x_i) < a. $$
    It's trivial that $f(x_{k-1},x_k) \leq a$.
    Thus $(x,y) \in U_m \circ U_m \circ U_m \subset U_{m-1}$, which
    implies that~$f(x,y) \leq 2^{-m+1} < 2a$.
\end{proof}
Theorem~\eqref{thm2.2} shows that a uniform space $(X,\mathcal{U})$ cannot be metrizable in general.
However, we can use a function similar to a metric to describe the closeness between two points in $X$.
This leads to the following definition of pseudometric.
\begin{definition}~\label{def2.2}
    Let $X$ be a nonempty set. A function $d:X \times X \rightarrow \RR_+$ is called a \textbf{pseudometric} on $X$ if it satisfies the following conditions:
    \begin{enumerate}
        \item For any $x \in X$, $d(x,x)=0$;
        \item For any $x,y \in X$, $d(x,y)=d(y,x)$;
        \item For any $x,y,z \in X$, $d(x,z) \leq d(x,y) + d(y,z)$.
    \end{enumerate}
    The pair $(X,d)$ is called a \textbf{pseudometric space}.
\end{definition}
It's possible that $\mathcal{U}$ can be induced by a family of pseudometrics.
\begin{theorem} \label{thm2.3}
    Suppose that $(X,\mathcal{U})$ is a uniform space.
    Let $P$ be the set of all uniformly continous pseudometrics on $X$.
    Then the uniform structure $\mathcal{U}$ is induced by the family $P$.
    Moreover, $P$ is the largest family of pseudometrics on $X$ inducing the uniform structure~$\mathcal{U}$.
\end{theorem}
We call the family $P$ mentioned in Theorem~\eqref{thm2.3} a \textbf{gage}.

\begin{proof}
    Since $d=0$ is an element of $P$, $P$ is nonempty.
    We write $\mathcal{U}_P$ for the uniform structure induced by $P$
    and will prove that $\mathcal{U}=\mathcal{U}_P$.
    For any $d \in P$ and $r > 0$, define
    $$ V_{d,r} = \{(x,y) \in X \times X: d(x,y) < r\}. $$
    By the uniform continuity of $d$, we have $V_{d,r} \in \mathcal{U}$.
    In addition, we know that $\mathcal{U_P}$ is the filter generated by the set
    $$ \mathcal{A}=\{V_{d,r}: d \in P, r > 0\}. $$
    Thus we have $\mathcal{U}_P \subset \mathcal{U}$.
    To prove the converse inclusion, we only need to prove~$\mathcal{A} \vdash \mathcal{U}$.
    For any $U \in \mathcal{U}$, choose $U_1=U \cap U^{-1}$. 
    There exists a sequence $\{U_n\}$ such that $U_n$ is symmetric and
    $$ U_{n} \circ U_{n} \circ U_{n} \subset U_{n-1}, \quad n \geq 2, $$
    where we set $U_0 = X \times X$.
    The same as the proof of Theorem~\eqref{thm2.2}, there exists a pseudometric $d$ on $X$ such that
    $$ V_{d,\sfrac{1}{2^{n+1}}} \subset U_n \subset V_{d,\sfrac{1}{2^{n-1}}} \text{ for all } n \in \NN. $$
    This also implies that $d$ is uniformly continuous.
    Thus $d \in P$ and $ \mathcal{A} \vdash U $.
\end{proof}
An interesting question is that, given a topological space $(X,\tau)$,
does there exist a uniform structure $\mathcal{U}$ on $X$ such that $\tau$ is the uniform topology induced by $\mathcal{U}$?
We can answer the question by using Theorem~\eqref{thm2.3}.
\begin{corollary} \label{cor2.4}
    Let $(X,\tau)$ be a topological space.
    Then there exists a uniform structure $\mathcal{U}$ on $X$ such that $\tau$ is the uniform topology induced by $\mathcal{U}$ if and only if $(X,\tau)$ is completely regular.   
\end{corollary}
\begin{proof}

\end{proof}

%%%%%%%%%%%%%%%%%%%%%%%%%%%%%%%%%%%%%%%%%%%%%%%%%%%%%%%%%%%%%%%%%%%%%%%%%%%%%%%%%%%%%%%%%%%%%%%%%%%%
%% References
% Include references only if there are citations.
\ifnum\value{cite}>0
    \small
    \bibliography{\bibfile}
    \bibliographystyle{plain}
\fi

%% The end
\end{document}